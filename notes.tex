\documentclass{article}
\usepackage[utf8]{inputenc}
\usepackage[margin=0.75in]{geometry} % lots more margin
\pagenumbering{gobble} % ignore page numbers

\usepackage{titling}
\setlength{\droptitle}{-0.75in}

\setlength{\parindent}{0cm}

\usepackage{enumitem}
\usepackage{graphicx}
\usepackage{amsmath}
\usepackage{amsfonts}
\usepackage{hyperref} % for nice looking urls
\usepackage{booktabs} % for making tables
\usepackage{amssymb}
\usepackage{listings}
\usepackage{graphicx}
\usepackage{caption}
\usepackage{subfigure}
\usepackage{multicol}

\usepackage{titlesec}
\usepackage{float}
\titleformat*{\section}{\large\bfseries}

\usepackage{tikz}
\usetikzlibrary{arrows}

\begin{document}


\section*{PSYC 402: Structural Equation Modelling}

By Jack Zhou

\begin{multicols*}{2}

% Any model can be thought of as a set of variables, of various sorts, to a set of parameter matrices of various sorts, 

% and a set of equations, that express the relationships between the variables.

% models are about explanation

\section{Linear Algebra Review}

TBD

\section{Three classes of models}

Structural Equations can be classified by 3 different classes.

\subsection{Path Model}

A path models can also be referred to as a multivariate regression model for observed cariables, or a causal model.

A path model can contain $p$ dependent (\textit{endogenous}) variables and $q$ independent (\textit{exogenous}) variables, denoted as the following:

\begin{itemize}
    \item $\mathbf{\vec{y}}$: a $p \times 1$ vector of \textit{endogenous} observed variables
    \item $\mathbf{\vec{x}}$: a $q \times 1$ vector of \textit{exogenous} observed variables
    \item $\mathbf{\vec{\zeta}}$: a $p \times 1$ vector of residuals
\end{itemize}

As suggested by the match of dimensionality between $\mathbf{\vec{y}}$ and $\mathbf{\vec{\zeta}}$, each of the entries in $\mathbf{\vec{y}}$ corresponds to a value in $\mathbf{\vec{\zeta}}$ with the same index. That is, for each dependent variable $\mathbf{\vec{y}}_i$, we have a corresponding residual $\mathbf{\vec{\zeta}}_i$ that the model is unable to account for.

It is important to note that path models do not contain any latent (unobserved) variables, as we can see in the following equation: 

\[\mathbf{\vec{y}} = [\mathbf{B} \mathbf{\vec{y}} + \mathbf{\Gamma} \mathbf{\vec{x}}] + \mathbf{\vec{\zeta}}\]

The general interpretation is that with this model, the set of endogenous variables are to be described in terms of a sum between the structural model $[\mathbf{B} \mathbf{\vec{y}} + \mathbf{\Gamma} \mathbf{\vec{x}}]$ and residuals $\mathbf{\vec{\zeta}}$.

The model consists of two parts, the $\mathbf{B} \mathbf{\vec{y}}$ term describes the regression of endogenous variables onto themselves, and the $\mathbf{\Gamma} \mathbf{\vec{x}}$ term describes the regression of exogenous variables onto the endogenous variables, in combined effort to best account for the endogenous variables.

One glaring issue is that in order to solve for $\mathbf{\vec{y}}$, it should only be on one side of the equation, not both. We can address this with algebra:

\[
\begin{aligned}
    \mathbf{\vec{y}} &= [\mathbf{B} \mathbf{\vec{y}} + \mathbf{\Gamma} \mathbf{\vec{x}}] + \mathbf{\vec{\zeta}} \\
    \mathbf{\vec{y}} - \mathbf{B} \mathbf{\vec{y}} &= \mathbf{\Gamma} \mathbf{\vec{x}} + \mathbf{\vec{\zeta}} \\
    (I - \mathbf{B}) \mathbf{\vec{y}} &= \mathbf{\Gamma} \mathbf{\vec{x}} + \mathbf{\vec{\zeta}} \\
    \mathbf{\vec{y}} &= (I - \mathbf{B})^{-1}(\mathbf{\Gamma} \mathbf{\vec{x}} + \mathbf{\vec{\zeta}}) \\
\end{aligned}    
\]

The equation $\mathbf{\vec{y}} = (I - \mathbf{B})^{-1}(\mathbf{\Gamma} \mathbf{\vec{x}} + \mathbf{\vec{\zeta}})$ is known as the reduced form, which is easier to work with mathematically.

Below are descriptions to all the matrices in a path model:
\begin{itemize}
    \item $\mathbf{B}$ : $p \times p$ regression coefficients for $\mathbf{\vec{y}}$
    \item $\mathbf{\Gamma}$ : $q \times q$ regression coefficients for $\mathbf{\vec{x}}$
    \item $\mathbf{\Phi}$ : $q \times q$ covariance matrix of $\mathbf{\vec{x}}$
    \item $\mathbf{\Psi}$ : $p \times p$ covariance matrix of $\mathbf{\vec{\zeta}}$
\end{itemize}

Note that $\mathbf{\Phi}$ and $\mathbf{\Psi}$ are matrices that aren't seen in the model. 

% TODO: move this to linalge review
For any covariance matrix $\mathbf{M}$, any $\mathbf{M}_{ii}$ entry is the variance of the $i$th variable and any $\mathbf{M}_{ij}$ entry is the covariance between the $i$th and $j$th variable.

$\mathbf{\Phi}$ can be considered a parameter for $\mathbf{\vec{y}}$. Therefore, we consider it as a starting point.

Often times we see correlations between residuals (ex. in longitudinal models), so $\mathbf{\Psi}$ does not necessarily have to be diagonal.

\subsubsection{Kinds of Path Models}

\begin{itemize}
    \item \textbf{Standard Multiple Regression}

    Suppose we have an endogenous variable $y_1$, its corresponding $\zeta_1$, and a set of exogenous variables $\{x_1,...,x_q\}$. A Standard Multiple Regression can be seen as a digraph with the following properties:
    
    \begin{itemize}
        \item An edge $(\zeta_1, y_1)$ exists; the residual impacts the endogenous variable
        \item For all $x_i$ in the set of exogenous variables, an edge $(x_i, y_1)$ exists; each of the exogenous varibales impacts the endogenous variable
        \item For all $x_i \neq x_j$, an edge $(x_i, x_j)$ exists to indicate correlations between exogenous variables
        
    \end{itemize}
    
    Let $\mathbf{\vec{x}}$ be a $p \times 1$ vector of exogenous variables, $\Gamma$ be a $1 \times q$ vector of regression coefficients for $\mathbf{\vec{x}}$, then such graph can be written as an equation as follows: \[y_1 = \Gamma \mathbf{\vec{x}} + \zeta_1\]

    \item \textbf{Standard Multivariate Regression}
    
    In this case, there are greater than $p > 1$ endogenous variables as well as $q > 1$ exogenous variables. 
    
    The setting is similar to the Standard Multiple Regression case, but there is more than one $y_i$, but no edges between any $y_i \neq y_j$ (no correlation between any endogenous variables).

    In addition to the notation used in standard multiple regression, we let $\mathbf{\vec{y}}$ be the $p \times 1$ vector of endogenous variables, and $\mathbf{B}$ be a $p \times p$ matrix as defined before. As the model asserts that there are no correlations between any endogenous variables, $\mathbf{B}$ is a zero matrix, and the covariance matrix of $\mathbf{\vec{y}}$, $\mathbf{\Psi}$, is taken to be diagonal.

    \item \textbf{First Order Autoregressive Time Series}
    
    A chain of measurements are collected in sequential order, imposing a temporal structure.

    The first measurement is considered exogenous, so we would name it $x_1$. From there on, each measurement that comes after are all endogenous variables $y_i$ with residuals $\zeta_i$. 
    
    Each variable is linearly dependent on the variable before, causing a chain of dependencies.
\end{itemize}

\end{multicols*}
\end{document}